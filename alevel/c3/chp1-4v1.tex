\documentclass[11pt]{exam}
\RequirePackage{amssymb, amsfonts, amsmath, latexsym, verbatim, xspace, setspace, mathrsfs}
\usepackage{amsmath,amsthm,amssymb,amsfonts, color, hyperref, graphicx}
\RequirePackage{tikz, pgflibraryplotmarks}
\usepackage[margin=1in]{geometry}
\usepackage{graphicx}

% design question/answer space
\setlength\linefillheight{.25in}
%\definecolor{FillWithLinesColor}{gray}{0.8}
\newcommand{\sol}[1]{\color{gray}\fillwithlines{#1}\color{black}}

%Shortcut symbols
\newcommand{\N}{\mathbb{N}}
\newcommand{\Z}{\mathbb{Z}}
\newcommand{\Q}{\mathbb{Q}}
\newcommand{\R}{\mathbb{R}}
\newcommand{\e}[1]{\,\mathrm{e}^{#1}}
%\newcommand{\ln}[1]{\mathrm{ln}( #1 )}
 
\newenvironment{problem}[2][Problem:]{\begin{trivlist}
\item[\hskip \labelsep {\bfseries #1}\hskip \labelsep {\bfseries #2}]}{\end{trivlist}}

\newenvironment{claim}[2][Claim:]{\begin{trivlist}
\item[\hskip \labelsep {\bfseries #1}\hskip \labelsep {\bfseries #2}]}{\end{trivlist}}

\newenvironment{defn}[2][Definition:]{\begin{trivlist}
\item[\hskip \labelsep {\bfseries #1}\hskip \labelsep {\bfseries #2}]}{\end{trivlist}}

% Here's where you edit the Class, Exam, Date, etc.
\newcommand{\class}{Core Mathematics C3 Advanced}
\newcommand{\term}{Winter 2016}
\newcommand{\examnum}{C3 Chapter 1 to 4}
\newcommand{\examdate}{10/10/16}
\newcommand{\timelimit}{60 minutes}

% For an exam, single spacing is most appropriate
\singlespacing
% \onehalfspacing
% \doublespacing

% For an exam, we generally want to turn off paragraph indentation
\parindent 0ex

\begin{document} 

% These commands set up the running header on the top of the exam pages
\pagestyle{headandfoot}
\firstpageheader{}{}{}
\runningheader{\class}{\examnum}{\examdate}
\runningheadrule
\firstpagefooter{}{}{}
\runningfooter{}{}{Page \thepage\ of \numpages}
\runningfootrule

 % set up the frontpage header
\begin{flushright}
\begin{tabular*}{\textwidth}{l @{\extracolsep{\fill}} r @{\extracolsep{6pt}} l}
\textbf{\class} & \textbf{Name:} & \makebox[2in]{\hrulefill}\\
\textbf{\term} &&\\
\textbf{\examnum} & Parent: & \makebox[2in]{\hrulefill}\\
\textbf{\examdate} &&\\
\textbf{Time Limit: \timelimit} & Teacher & \makebox[2in]{\hrulefill}
\end{tabular*}\\
\end{flushright}
\rule[1ex]{\textwidth}{2pt}




\begin{minipage}[t]{3.7in}
\vspace{0pt}
\begin{itemize}

\item \textbf{DO NOT open the exam booklet until you are told to begin. You should write your name at the top and read the instructions.}

\vfill

\item Organize your work, in a reasonably neat and coherent way, in
the space provided. If you wish for something to not be graded, please strike it out neatly. I will grade only work on the exam paper, unless you clearly indicate your desire for me to grade work on additional pages.

\item You may use any results from class, homework or the text, but you must cite the result you are using. You must prove everything else.

\item You needn't spend your time rewriting definitions or axioms on the exam.

\item 
%You may use the text, my class notes and/or any notes and study guides you have created.
Candidates may use any \textbf{calculator} allowed by the regulations of the Joint
Council for Qualifications. Calculators must not have the facility for symbolic
algebra manipulation or symbolic differentiation/integration, or have
retrievable mathematical formulae stored in them.

\end{itemize}


\end{minipage}
\hfill
\begin{minipage}[t]{2.3in}
\vspace{0pt}
%\cellwidth{3em}
\gradetablestretch{2}
%Uncomment this line to make the table display 100 as the total no matter what. This is good for tests with an ommit question.
%\settabletotalpoints{100}
%\vqword{Problem}
\addpoints % required here by exam.cls, even though questions haven't started yet.	
\gradetable[v][questions]  % Use [pages] to have grading table by page instead of question

\end{minipage}

\begin{itemize}
\item
There are \numquestions\ questions in this question paper. The total mark for this paper is \numpoints.
There are \numpages\ pages in this question paper. 



\item When you have completed your test, hand it to me and go have a great week!

%\item There is a single bonus problem at the end of the test. It would be best to work first on the main test as this problem is only worth 5 points and will be graded strictly.

\end{itemize}

%%%%%%%%%%%%%%%%%%%%%%%%%%%%%%%%%%%%%%%%%%%%%%%%%%%%%%%%%%%%%%%%%%%%%%%%%%%%%%%%%%%%%%%%%
%%%%%%%%%%%%%%%%%%%%%%%%%%%%%%%%%%%%%%%%%%%%%%%%%%%%%%%%%%%%%%%%%%%%%%%%%%%%%%%%%%%%%%%%%
%%%%%%%%%%%%%%%%%%%%%%									%%%%%%%%%%%%%%%%%%%%%%%%%%%%%%%%%
%%%%%%%%%%%%%%%%%%%%%%		QUESTIONS Start Now			%%%%%%%%%%%%%%%%%%%%%%%%%%%%%%%%%
%%%%%%%%%%%%%%%%%%%%%%									%%%%%%%%%%%%%%%%%%%%%%%%%%%%%%%%%
%%%%%%%%%%%%%%%%%%%%%%%%%%%%%%%%%%%%%%%%%%%%%%%%%%%%%%%%%%%%%%%%%%%%%%%%%%%%%%%%%%%%%%%%%
%%%%%%%%%%%%%%%%%%%%%%%%%%%%%%%%%%%%%%%%%%%%%%%%%%%%%%%%%%%%%%%%%%%%%%%%%%%%%%%%%%%%%%%%%

\newpage 
\begin{questions}
\addpoints
%%% QUESTION %%%
\question[3] 
Use algebraic long division to divide \(x^3 + 2x^2 - x + 19\) by \(x+4\).
\sol{\fill}

%%% QUESTION %%%
\question[4] 
Write \(2x^3 + 8x^2 + 7x + 8\) in the form \( (A x^2 + B x +C)(x+3) + D\).
Using your answer, state the result when \(2x^3 + 8x^2 + 7x + 8\) is divided by \((x+3)\).
\sol{\fill}

\newpage 
\addpoints

%%% QUESTION %%%
\question 
	Here is a map, \[f(x) = \frac{5}{2x+1}\]
\begin{parts}
\part[2] 
	Evaluate \(f(0)\) and \(f(\frac{1}{2})\).
\part[3] 
	Draw the mapping diagram for the domain \( \{x \in \mathbb{N}, x<6\}\) and list the range.
\part[2] 
	Is the map a function for the domain \(x \in \mathbb{Z}\), why?
\part[2] 
	Is the map a function for the domain \(x \in \mathbb{R}\), why?
\end{parts}
\sol{2in}
\vfill
\sol{2.5 in}

\newpage 
\addpoints

%%% QUESTION %%%
\question
\begin{parts}
\part[2]
	Sketch the graph of the function
	\[
	f(x) =
	\begin{cases}
		x^2 - 2 & -2 < x<2 \\
	    2 & \text{otherwise}
	\end{cases}
	\]
\part[2]
	State the range of the function.
\end{parts}
\vfill
\sol{\fill}
\newpage
\addpoints

%%% Question %%%
\question
	\[
	f(x) = x+4, \text{  and  } g(x)=\frac{3}{x+1}, \, x>0 
	\]
\begin{parts}
\part[1]
	Find \(f^{-1}(x)\).
\part[2]
	Find \(g^{-1}(x)\).
\part[2]
	\(f^{-1}g^{-1}(x)\).
\part[2]
	Find \(gf(x)\).
\part[2]
	Find the inverse of \(gf(x)\). What do you notice?
\end{parts}
\sol{\fill}

\newpage
\addpoints

%%% Question %%%
\question
Sketch the graphs of the following, labelling key points and asymptotes:
\begin{parts}
\part[3]
	\(y = 2 - 4\e{x+1}\)
\part[3]
	\(y = \ln{(x+5)} + 1\)
\end{parts}
\vfill

\newpage
\addpoints

%%% Question %%%
\question
Solve the following equations, giving your solutions as exact values:
\begin{parts}
\part[3]
	\(2\ln{x} - \ln{(2x)} = 2\)
\part[3]
	\(\e{8x}-\e{4x}-6=0\)
\end{parts}
\sol{\fill}

\newpage
\addpoints

%%% Question %%%
\question
	The graph of the function:
	\[
    y=\sin{3x} + 3x, \quad 0<x<\pi
    \]
    meets the line \( y=1 \) when \(x=a\).
\begin{parts}
\part[4]
	Show that \(0.1<a<0.2\).
\part[2]
	Show that the equation:
    \[
    \sin{3x} + 3x = 1
    \]
    can be written as:
    \[
    x=\frac{1}{3}(1-\sin{3x})
    \]
\part[2]
	Starting with \(x_0 = 0.2\), use the iteration:
    \[
    x_{n+1}=\frac{1}{3}(1-\sin{3x_n})
    \]
    to find \(x_4\) to 3 d.p.
\part[3]
	Use a suitable interval to verify that your answer to part (c) is correct to 3 decimal places.
\end{parts}
\sol{\fill}

\end{questions}
%%%%%%%%%%%%%%%%%%%%%%%%%%%%%%%%%%%%%%%%%%%%%%%%%%%%%%%%%%%%%%%%%%%%%%%%%%%%%%%%%%%%%%%%%
%%%%%%%%%%%%%%%%%%%%%%%%%%%%%%%%%%%%%%%%%%%%%%%%%%%%%%%%%%%%%%%%%%%%%%%%%%%%%%%%%%%%%%%%%
%%%%%%%%%%%%%%%%%%%%%%									%%%%%%%%%%%%%%%%%%%%%%%%%%%%%%%%%
%%%%%%%%%%%%%%%%%%%%%%		QUESTIONS End Now			%%%%%%%%%%%%%%%%%%%%%%%%%%%%%%%%%
%%%%%%%%%%%%%%%%%%%%%%									%%%%%%%%%%%%%%%%%%%%%%%%%%%%%%%%%
%%%%%%%%%%%%%%%%%%%%%%%%%%%%%%%%%%%%%%%%%%%%%%%%%%%%%%%%%%%%%%%%%%%%%%%%%%%%%%%%%%%%%%%%%
%%%%%%%%%%%%%%%%%%%%%%%%%%%%%%%%%%%%%%%%%%%%%%%%%%%%%%%%%%%%%%%%%%%%%%%%%%%%%%%%%%%%%%%%%
\end{document}
